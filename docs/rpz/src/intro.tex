\begin{center}
    \textbf{ВВЕДЕНИЕ}
\end{center}
\addcontentsline{toc}{chapter}{ВВЕДЕНИЕ}

Кривые Безье и тела вращения являются элементами в компьютерной графике и проектировании, используемыми для создания плавных и точных поверхностей объектов.

Кривая Безье — тип кривой, предложенный в 60-х годах XX века независимо друг от друга Пьером Безье из автомобилестроительной компании «Renault» и Полем де Кастельжо из компании «Citroen», где применялись для проектирования кузовов автомобилей. Кривая Безье является частным случаем многочленов Бернштейна, описанных русским математиком Сергеем Натановичем Бернштейном в 1912 году. Впоследствии это открытие стало одним из важнейших инструментов систем автоматизированного проектирования и программ компьютерной графики~\cite{bezier}.

\textbf{Целью работы}: разработка программного обеспечения, которое позволяет пользователю генерировать тела вращения с помощью кривой Безье, выбирать цвет тела вращения, расположение источника света и камеры.

Для достижения поставленной цели необходимо решить следующие
задачи:

\begin{itemize}
    \item[---] изучение методов генерации кривой Безье и тел вращения;
    \item[---] анализ существующих алгоритмов создания кривой Безье и тел вращения;
    \item[---] выбор подходящих алгоритмов для решения поставленной задачи;
    \item[---] проектирование архитектуры и графического интерфейса программы;
    \item[---] реализация структур данных и алгоритмов для работы с кривой Безье и телами вращения;
    \item[---] описание структуры разрабатываемого ПО;
    \item[---] написание программы и тестирование;
    \item[---] исследование производительности программы при работе с телами вращения.
\end{itemize}